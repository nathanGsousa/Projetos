\documentclass[12pt,oneside]{abntex2}
\usepackage{mathptmx}
\usepackage[brazil]{babel}
\usepackage[utf8]{inputenc}
\usepackage[T1]{fontenc}
\usepackage{graphicx}
\usepackage{amsmath, amssymb}
\usepackage{lipsum}
\usepackage{setspace}
\usepackage{indentfirst}
\usepackage{setspace}
\usepackage{enumerate}
\usepackage{tocloft}
\usepackage{float}
\usepackage{hyperref}
\usepackage{geometry}
\geometry{a4paper, top=3cm, bottom=2cm, left=3cm, right=2cm}
\setcounter{section}{0} 
\renewcommand{\thesection}{\arabic{section}} 
\usepackage{placeins}



\title{Análise Econômica do comportamento da Inflação entre o período de 2012 à 2025 em Regressão Linear}
\author{Nathan Pereira Gomes de Sousa \\ José João Victor De Oliveira Lima Almeida }
\date{04 de Setembro de 2025}
\local{Campina Grande - PB}
\instituicao{Universidade Federal de Campina Grande \\
Curso de Bacharelado em Ciências Econômicas}
\tipotrabalho{Atividade Avaliativa}

% Espaçamento 1,5
\OnehalfSpacing

\begin{document}

\begin{center}
{\textbf{\fontsize{14}{16}\selectfont 
UNIVERSIDADE FEDERAL DE CAMPINA GRANDE\\[1ex]
CENTRO DE HUMANIDADES\\[1ex]
UNIDADE ACADÊMICA DE ECONOMIA E FINANÇAS}}
\end{center}

% Capa
\imprimircapa

% Folha de rosto
\imprimirfolhaderosto*

% Sumário
\tableofcontents*
\newpage

\section{\textbf{Introdução}}

Esta atividade tem como objetivo aplicar conceitos de regressão linear para investigar o comportamento da inflação no Brasil, por meio da análise empírica da Curva de Phillips aceleracionista ou moderna com expectativas racionais.

O estudo da macroeconomia concentra-se na análise de variáveis que afetam o funcionamento da economia como um todo. Dentre essas variáveis, destacam-se os índices de inflação e as taxas de desemprego, que são alvo de constantes estudos, especialmente pela ligação observada entre elas. Um dos modelos teóricos mais relevantes nesse contexto é a Curva de Phillips, que sugere a existência de uma correlação negativa entre inflação e desemprego em determinados períodos históricos (BLANCHARD, 2001).

A lógica por trás dessa relação está associada ao comportamento das empresas e trabalhadores. Quando o nível de atividade econômica está aquecido e o desemprego está baixo, os salários tendem a subir, o que, por sua vez, pode pressionar os preços para cima, gerando inflação. Por outro lado, quando há alto desemprego, a demanda por bens e serviços enfraquece, os salários crescem mais lentamente e a inflação tende a diminuir (BLANCHARD, 2001).

\subsection*{\textbf{Teorias econômicas e evidências empíricas}}

A origem da Curva de Phillips remonta ao trabalho de A. W. Phillips (1958), que, ao analisar dados do Reino Unido entre 1861 e 1957, identificou uma relação inversa entre inflação e desemprego. Posteriormente, Samuelson e Solow (1960) confirmaram resultados semelhantes para os Estados Unidos, com dados de 1900 a 1960. Essa descoberta teve grande influência na formulação de políticas econômicas, pois indicava que governos poderiam escolher entre inflação baixa com desemprego alto ou inflação alta com desemprego baixo.  

Na década de 1970, porém, essa relação quebrou. Nos Estados Unidos, assim como na maioria dos países da OCDE, verificou-se inflação alta acompanhada de desemprego elevado, o que contradizia explicitamente a curva de Phillips original. Uma relação voltou a aparecer, mas sob a forma de associação entre a taxa de desemprego e a variação da taxa de inflação (Blanchard, 2001).  

Por que a curva de Phillips original desapareceu? Porque os fixadores de salários mudaram a forma de formar suas expectativas em relação à inflação. Essa mudança decorreu da própria transformação no comportamento da inflação: esta se tornou mais persistente, aumentando a probabilidade de que um ano de inflação alta fosse seguido por outro também elevado. Assim, os agentes passaram a levar em conta essa persistência ao formar suas expectativas. Essa alteração na formação de expectativas modificou a natureza da relação entre desemprego e inflação (Blanchard, 2001).  

Blanchard (2001) ressalta que a relação negativa prevista pela curva de Phillips é observada quando, no eixo horizontal, coloca-se a taxa de desemprego e, no eixo vertical, as variações da taxa de inflação. Nesse caso, a taxa de desemprego determina a aceleração da inflação.  

A formulação inicial da curva é dada por:

\[
\pi = (u - u_n)
\]

onde:
\begin{itemize}
    \item $\pi$ representa a inflação;
    \item $(u - u_n)$ representa a diferença entre o desemprego efetivo e o desemprego natural.
\end{itemize}

A diferença $(u - u_n)$ capta o desvio do desemprego em relação ao seu nível natural. Quando o desemprego efetivo $u$ está abaixo do desemprego natural $u_n$, a pressão sobre os preços tende a ser positiva, resultando em inflação. Por outro lado, quando o desemprego efetivo é superior ao nível natural, a pressão é negativa, indicando tendência de desaceleração inflacionária ou até deflação.  

Essa relação expressa o núcleo da Curva de Phillips, que sugere um trade-off de curto prazo entre inflação e desemprego. Entretanto, a experiência empírica mostrou que esse mecanismo não se mantinha estável ao longo do tempo, sobretudo quando a inflação passou a se tornar persistente.  

Enquanto a inflação permanecia relativamente baixa e pouco persistente, trabalhadores e empresas tendiam a assumir uma expectativa constante para a inflação futura, muitas vezes desconsiderando a inflação passada. Esse foi o contexto inicial observado, em que o desemprego $u$ girava em torno de níveis próximos ao natural e as expectativas eram praticamente fixas.  

Contudo, à medida que a inflação começou a se mostrar mais duradoura, especialmente a partir da década de 1970, trabalhadores e empresas passaram a ajustar suas expectativas com base na inflação passada. Nesse contexto, se a inflação tivesse sido elevada no período anterior, esperava-se que também fosse elevada no período seguinte. Esse processo levou à incorporação das expectativas na formulação da Curva de Phillips, resultando na chamada **Curva de Phillips aumentada pelas expectativas**:

\[
\pi = \pi^e - \phi (u_t - u_n)
\]

onde:
\begin{itemize}
    \item $\pi^e$: representa a inflação esperada;
    \item $\phi > 0$: é o parâmetro que mede a sensibilidade da inflação em relação ao desvio do desemprego.
\end{itemize}

Assim, a versão aumentada ou aceleracionista reconhece que não basta considerar apenas o hiato do desemprego, mas também a forma como os agentes formam expectativas sobre a inflação futura. Essa formulação indica que uma taxa de desemprego baixa leva a um aumento da taxa de inflação e, consequentemente, a uma aceleração do nível de preços. 

\subsection*{\textbf{Expectativas adaptativas e racionais}}

Uma vez que a expectativa de inflação influencia o trade-off entre inflação e desemprego no curto prazo, é crucial compreender como os agentes formam suas expectativas. Inicialmente, admite-se a hipótese de expectativas adaptativas, segundo a qual a inflação esperada depende da inflação recentemente observada. Nesse caso, trabalhadores e empresas ajustam suas previsões de forma gradual, levando em conta a inflação passada. Embora plausível, esse pressuposto é considerado simples demais para capturar todas as circunstâncias relevantes.

Uma abordagem alternativa e mais sofisticada é a das expectativas racionais. Nessa formulação, presume-se que os agentes fazem o melhor uso possível das informações disponíveis ao tomar decisões, incluindo dados sobre a condução das políticas monetária e fiscal. Assim, a inflação esperada $\pi_t^e$ não depende apenas do comportamento passado da inflação, mas também da credibilidade e do regime de política econômica vigente. Os erros do passado deixam de influir diretamente nas expectativas do presente, e os agentes não incorrem em erros sistemáticos, isto é, os erros de diferentes períodos não são autocorrelacionados.

Na versão fraca da hipótese de expectativas racionais, os agentes apenas utilizam todas as informações disponíveis, sem vínculos adicionais. Já na versão forte, assume-se que os agentes, em média, acertam o valor efetivo da variável. Formalmente:

\[
E(e_t) = 0 \quad \text{e} \quad \text{Cov}(e_t, e_{t-1}) = 0,
\]

em que $E$ denota a esperança matemática (média esperada), $\text{Cov}$ representa a covariância e $e_t$ é o erro de previsão.  

Assumir expectativas racionais traz implicações relevantes para a análise econômica. Considere, por exemplo, o caso em que os salários sejam fixos devido a contratos de trabalho. Se houver uma expansão monetária inesperada, a demanda agregada se desloca, elevando o nível de preços e reduzindo o salário real. Isso torna o trabalho mais barato, induzindo as empresas a contratar mais mão de obra, elevando o produto. Por outro lado, se a expansão monetária já era antecipada, trabalhadores e empresas podem incluir cláusulas de correção automática nos contratos, ajustando salários nominais. Nesse caso, a expansão da oferta monetária eleva simultaneamente demanda e oferta agregada, neutralizando os efeitos sobre emprego e produto.

Dessa forma, a Curva de Phillips com expectativas racionais pode ser expressa como:

\[
\pi_t = \pi_t^e - \phi (u_t - u_n) + v_t,
\]

onde:
\begin{itemize}
    \item $\pi_t$ é a inflação no período $t$;
    \item $\pi_t^e$ a inflação esperada;
    \item $u_t$ o desemprego efetivo;
    \item $u_n$ o desemprego natural;
    \item $\phi > 0$ mede a sensibilidade da inflação ao desvio do desemprego; 
    \item $v_t$ representa choques aleatórios.  
\end{itemize}

Como argumenta Sargent (2011), a persistência inflacionária observada não decorre de uma “força inercial” da inflação em si, mas das políticas econômicas que alimentam expectativas futuras. Dessa perspectiva, a inflação pode ser reduzida de forma mais rápida e menos custosa do que preveem modelos baseados em expectativas adaptativas, desde que a política adotada seja crível (Mankiw, 2011).  

Nessa hipótese, o trade-off de curto prazo desaparece, mostrando como a credibilidade da política econômica é central para a Curva de Phillips aumentada pelas expectativas.

\section{\textbf{Metodologia}}

A relação entre inflação e desemprego tem sido amplamente discutida na teoria macroeconômica, nesse estudo a partir da formulação da Curva de Phillips e a sua versão aceleracionista em que ela descreve uma interdependência negativa entre a variával explica inflação (IPCA) e a taxa de desemprego sugerindo que, em determinados contextos, taxas mais baixas de desemprego estão associadas a taxas mais altas de inflação, e vice-versa, assim como a variavel expectativa de inflação e a taxa básica de juros do Brasil entre 2012 e 2025.

\subsection{\textbf{Escolha das Variáveis Econômicas}}

Ás variáveis escolhidas foram Índice de Preço ao Consumidor Amplo (IPCA) usado para mensurar a inflação do Brasil, o nível de desemprego na Economia Brasileira, dessazonalizada pelo Instituto de Economia Aplicada (IPEA) a taxa básica de juros over, a expectativa de inflação do Boletim FOCUS como proxy para a expectativa de inflação do mercado, no periodo entre janeiro de 2012 (01/2012) até maio de 2025 (05/2025).

A escolha dessas variáveis se justifica pela relevância da Curva de Phillips, que propõe uma relação negativa entre desemprego e inflação em determinados contextos macroeconômicos e observar a relevancia das expectativas de inflação. 

O IPCA é o principal índice de inflação utilizado pelo Banco Central do Brasil para a formulação da política monetária, refletindo a variação de preços de uma cesta de bens e serviços consumidos pelas famílias com rendimentos entre 1 e 40 salários mínimos. Já a taxa de desemprego é um dos principais indicadores do mercado de trabalho, refletindo a proporção da força de trabalho que está desocupada e procurando emprego. Ambas as variáveis são fundamentais para a compreensão do comportamento da economia e são amplamente utilizadas na formulação de políticas econômicas.

%Dessazonalizada

%Selic

%expect

\subsection*{\textbf{Relevância dos dados atuais}}

No contexto brasileiro, a análise da Curva de Phillips continua relevante, especialmente diante dos desafios macroeconômicos dos últimos anos, como os impactos da pandemia da COVID-19, choques nos preços internacionais, crises políticas e medidas de política monetária adotadas pelo Banco Central.

\subsection{\textbf{Descrição e Coleta de Dados}}

Os dados utilizados neste trabalho foram obtidos a partir de três fontes principais: Instituto de Pesquisa Econômica Aplicada (IPEA), Sistema Gerenciador de Séries Temporais (SGS) e o Sistema Expectativas de Mercado ambos do Banco Central do Brasil. As séries foram selecionadas de acordo com a disponibilidade, relevância e consistência metodológica para a análise da Curva de Phillips no contexto brasileiro.

Para a inflação, foi considerado o Índice Nacional de Preços ao Consumidor Amplo (IPCA), série \texttt{PRECOS12\_IPCAG12}, disponibilizada pelo IBGE, cuja frequência é mensal. O IPCA mede a variação dos preços de uma cesta de bens e serviços representativa do consumo das famílias brasileiras, abrangendo rendimentos entre 1 e 40 salários mínimos e aproximadamente 90\% das famílias residentes em áreas urbanas cobertas pelo \textit{Sistema Nacional de Índices de Preços ao Consumidor} (SNIPC).

Para a taxa de desemprego, adotou-se a série \texttt{PNADC12\_TDESOCMD12}, obtida no IPEA, que corresponde à taxa de desocupação das pessoas de 14 anos ou mais de idade, mensalizada e dessazonalizada. Essa série é derivada da PNAD Contínua do IBGE, com ajustes metodológicos descritos na Nota Técnica nº 62 do IPEA, cobrindo o período de 2012.01 até 2025.05. A dessazonalização foi realizada pelo método X-13 ARIMA, sem ajustes para feriados móveis ou \textit{outliers}.

Adicionalmente, incluiu-se a taxa de juros de referência, a Selic, série \texttt{BM12\_TJOVER12}, disponibilizada pelo Banco Central do Brasil. Trata-se da taxa de juros \textit{Over/Selic} acumulada no mês, calculada a partir das operações de um dia entre instituições financeiras, garantidas por títulos públicos, registrada no Sistema Selic.

Além das variáveis efetivas de inflação (IPCA), desemprego (PNAD Contínua/IPEA) e taxa de juros (Selic), este trabalho também incorpora as expectativas de inflação extraídas do \textit{Sistema de Expectativas de Mercado} do Banco Central do Brasil. Essas expectativas resultam do acompanhamento contínuo das projeções feitas por instituições financeiras, consultorias e demais agentes de mercado, sendo consolidadas no Boletim Focus.

O uso das expectativas de inflação tem como objetivo captar a dimensão prospectiva do comportamento dos agentes econômicos, refletindo como estes formam suas previsões sobre a dinâmica futura dos preços. Tal variável pode ser considerada uma \textit{proxy} para a inflação esperada, dado que não é observada diretamente na economia, mas inferida a partir de informações coletadas pelo Banco Central. 

Assim, a inclusão dessa medida é coerente com a literatura macroeconômica que ressalta o papel central das expectativas na determinação da inflação e na formulação da política monetária. Ao controlar para a inflação esperada, busca-se avaliar se a relação entre inflação observada e desemprego mantém-se consistente com a Curva de Phillips ou se passa a depender fortemente das expectativas dos agentes.

\medskip

O período de análise considerado neste estudo vai de janeiro de 2012 (01/2012) até maio de 2025 (05/2025), respeitando a interseção temporal entre as séries. Após a coleta, os dados foram tratados no software estatístico R: as séries foram convertidas para o formato de data adequado, inspecionadas quanto à presença de valores faltantes, harmonizadas em frequência e unificadas em um único \textit{dataframe}. Foram realizadas análises descritivas e gráficas, bem como estimativas econométricas (regressão linear), com o objetivo de investigar empiricamente a relação entre inflação e desemprego e avaliar a validade da Curva de Phillips no caso brasileiro.

\medskip
\noindent \textbf{Fontes:}  
Instituto de Pesquisa Econômica Aplicada (IPEA) – Taxa de desocupação; \\
Instituto Brasileiro de Geografia e Estatística (IBGE/SNIPC) – IPCA; \\
Banco Central do Brasil (BACEN) – Taxa Selic; \\
Banco Central do Brasil – Sistema de Expectativas de Mercado – Expectativas de inflação. \\

\textbf{Período:} Mensal, de Janeiro de 2012 (01/2012) a Maio de 2025 (05/2025).
\medskip

\subsection{\textbf{Modelo}}

Para tanto, será empregada a regressão linear múltipla, ferramenta estatística fundamental para investigar a relação entre uma variável dependente e múltiplas variáveis explicativas. No presente estudo, considera-se a taxa de inflação (IPCA) como variável dependente, enquanto a taxa de desemprego (IPEA/PNAD Contínua), a taxa básica de juros (Selic) e as expectativas de inflação (proxy de inflação esperada) são utilizadas como variáveis independentes. Dessa forma, pretende-se avaliar em que medida essas variáveis explicativas ajudam a compreender a dinâmica da inflação e se o

O modelo de regressão linear multipla é representado pela equação:

\[
y = \alpha + \beta_1 x_1 + \beta_2 x_2 + \ \beta_3 x_3  +\ \beta_4 x_4 + \varepsilon_i
\]

onde:
\begin{itemize}
    \item $y$ representa a variável dependente inflação;

    \item $x_1$ representa a variável independente desemprego;

    \item $x_2$ representa a variável independente taxa básica de juros mensal;

    \item $x_3$ representa a variável independente expectativas de inflação do mercado;

    \item $x_4$ representa a variável independente ;

    \item $\alpha$ é o intercepto da reta, também chamado de coeficiente linear, indicando o valor esperado de $y$ quando $x$=0;

    \item $\beta_i$ é o coeficiente angular, que representa a variação esperada em Y para cada unidade de variação em $x_i$;

    \item $\varepsilon_i$ é o termo de erro, que captura os efeitos de outras variáveis não incluídas no modelo.
\end{itemize}

A escolha por esse tipo de regressão se justifica por sua simplicidade, interpretabilidade e ampla aplicação em análises econômicas. Além disso, permite verificar se há uma associação estatisticamente significativa entre desemprego e inflação em determinados períodos históricos, contribuindo para a compreensão da aplicabilidade empírica da Curva de Phillips no Brasil

\section{\textbf{Análise dos Resultados}}





\begin{figure}[htbp]
    \centering
    \includegraphics[width=0.7\textwidth]{Phillips(2020-2025)plot.png}
    \caption{Curva de Phillips 2020-2025}
    \label{fig:minha-imagem}
\end{figure}

\section{\textbf{Conclusão}}

\section{\textbf{Referências bibliográficas}}
\setlength{\parindent}{0pt} % Remove o recuo da primeira linha
\setlength{\itemindent}{0pt} % Garante alinhamento
\setlength{\leftskip}{0pt}   % Garant


BACHA, Carlos José Caetano; LIMA, Roberto Arruda de Souza. \textbf{A curva de Phillips: teoria, evidência empírica e aplicabilidade à economia brasileira}. \textit{Pesquisa e Debate}, São Paulo, v. 15, n. 1(25), p. 131–162, 2004.



BANCO CENTRAL DO BRASIL. \textbf{Sistema Gerenciador de Séries Temporais – SGS.} Disponível em: \url{https://www3.bcb.gov.br/sgspub}. Acesso em: 27 ago. 2025.



BLANCHARD, Olivier. \textbf{\textit{Macroeconomia}}. 3. ed. São Paulo: Pearson, 2001.



GUJARATI, Damodar N. \textbf{\textit{Econometria básica}}. 4. ed. Rio de Janeiro: Elsevier, 2006.



MANKIW, N. Gregory. \textbf{\textit{Introdução à economia}}. 7. ed. São Paulo: Cengage Learning, 2021.



\end{document}
